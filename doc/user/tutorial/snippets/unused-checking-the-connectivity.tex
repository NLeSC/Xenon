\subsubsection{Checking the connectivity}

Before we can run the \texttt{CreatingXenon} example, we first have to make sure that we have access to a remote system. You'll need an account on the remote machine. For example, I have an account \texttt{jspaaks} on SURFsara's Lisa clustercomputer. Cluster computers typically have a dedicated machine (the so-called `headnode') that serves as the main entry point when connecting from outside the cluster. For Lisa, the headnode is located at \url{lisa.surfsara.nl}.

I can connect to Lisa's head node using the \texttt{ssh} command line program as follows:
\begin{lstlisting}[style=basic,style=bash,escapeinside={(*@}{@*)}]
# (my account on Lisa is called jspaaks)
ssh jspaaks@lisa.surfsara.nl
\end{lstlisting}

If this is the first time you connect to the remote machine, it will generally ask if you want to add the remote machine to the list of `known hosts'. For example, here's what the Lisa system tells me when I try to ssh to it:
\begin{lstlisting}[style=basic,style=bash,escapeinside={(*@}{@*)}]
The authenticity of host 'lisa.surfsara.nl (145.100.29.210)' can't be
established.
RSA key fingerprint is b0:69:85:a5:21:d6:43:40:bc:6c:da:e3:a2:cc:b5:8b.
Are you sure you want to continue connecting (yes/no)?
\end{lstlisting}
If I then type \texttt{yes}, it says\footnote{SURFsara publish RSA key fingerprints for their systems at \url{https://userinfo.surfsara.nl/systems/shared/ssh}. The number posted there should be the same as what you have in your terminal.}:
\begin{lstlisting}[style=basic,style=bash,escapeinside={(*@}{@*)}]
Warning: Permanently added 'lisa.surfsara.nl,145.100.29.210' (RSA) to
         the list of known hosts.

                             (*@\textit{<some content omitted>}@*)
\end{lstlisting}
and asks for my password.

The result of this connection is that you should now have a (hidden) directory \texttt{.ssh} in your \texttt{/home} directory, which should contain 3 files: \texttt{id\_rsa}, which contains your private RSA key(s); \texttt{id\_rsa.pub}, which contains your public RSA key(s); and \texttt{known\_hosts}, which contains a list of systems that you have successfully connected to in the past. \url{known_hosts} uses one line per known system, and each line begins with the following elements:
\begin{itemize}
\item{\texttt{1} a flag signifying that the third element (host name) is hashed using the SHA1 algorithm;}
\item{\texttt{x5PcOam9hhAjdF84++EKwodUNgQ} the (public) salt used to encrypt the host name;}
\item{\texttt{NK1rAZev7rV6JSTIdM3ymPpKlQ0}} the (hashed) host name;}
\item{key-value pairs, e.g. the RSA fingerprint of the Lisa system \url{ssh-rsa} \url{b0:69:85:a5:21:d6:43:40:bc:6c:da:e3:a2:cc:b5:8b}.}
\end{itemize}
Xenon uses \texttt{known\_hosts} to automatically connect to a (known) remote system, without having to ask for credentials every time.
